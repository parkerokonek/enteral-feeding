\documentclass[onecolumn, draftclsnofoot,10pt, compsoc]{IEEEtran}
\usepackage{graphicx}
\usepackage{url}
\usepackage{setspace}

\usepackage{geometry}
\geometry{textheight=9.5in, textwidth=7in}

\def \CapstoneTeamName{		Enteral Feeding Calculator Team}
\def \CapstoneTeamNumber{		90}
\def \GroupMemberOne{			Alison Jones}
\def \GroupMemberTwo{			Parker Okonek}
\def \CapstoneProjectName{		Volume-Based Enteral Feeding Calculator}
%\def \CapstoneSponsorCompany{	Company, Inc}
\def \CapstoneSponsorPerson{		Dr. Judy Davidson}

\def \DocType{
				Design Document
				}
			
\newcommand{\NameSigPair}[1]{\par
\makebox[2.75in][r]{#1} \hfil 	\makebox[3.25in]{\makebox[2.25in]{\hrulefill} \hfill		\makebox[.75in]{\hrulefill}}
\par\vspace{-12pt} \textit{\tiny\noindent
\makebox[2.75in]{} \hfil		\makebox[3.25in]{\makebox[2.25in][r]{Signature} \hfill	\makebox[.75in][r]{Date}}}}
% 3. If the document is not to be signed, uncomment the RENEWcommand below
%\renewcommand{\NameSigPair}[1]{#1}

%%%%%%%%%%%%%%%%%%%%%%%%%%%%%%%%%%%%%%%
\begin{document}
\begin{titlepage}
    \pagenumbering{gobble}
    \begin{singlespace}
        \hfill 
        \par\vspace{.2in}
        \centering
        \scshape{
            \huge CS Capstone \DocType \par
            {\large\today}\par
            \vspace{.5in}
            \textbf{\Huge\CapstoneProjectName}\par
            \vfill
            {\large Prepared for}\par
            {\Large\NameSigPair{\CapstoneSponsorPerson}\par}
            {\large Prepared by }\par
            Group\CapstoneTeamNumber\par
            \CapstoneTeamName\par 
            \vspace{5pt}
            {\Large
                \NameSigPair{\GroupMemberOne}\par
                \NameSigPair{\GroupMemberTwo}\par
            }
            \vspace{20pt}
        }
        \begin{abstract}%temp abstract until doc is complete.
        This document details the design components of the Enteral Feeding Calculator project. The purpose of this project is to increase efficiency and effectiveness among nurses during the tube feeding process. The document covers product importance, usage, stakeholders, and views. Our solution will be an application designed for either mobile device or computer interfaces with potential methods specified below.
        \end{abstract}     
    \end{singlespace}
\end{titlepage}
\newpage
\pagenumbering{arabic}
\tableofcontents
\clearpage






%=======================================================
\section{Overview}
%Alternatively we could just have an overview paragraph and no subsections.
\subsection{Purpose}
\subsection{Scope}
\subsection{Intended Audience}
\section{Definitions}
%enteral feeding
%other weird words?
%etc.


%=======================================================
\section{Project Context}
\subsection{Hardware}
\subsection{Software}

%=======================================================
\section{Design Description}
%Stakeholders, Views, Viewpoints, Rationale
\subsection{Design Stakeholders}
\subsection{Design View}
\subsubsection{Users}
\subsubsection{others?}%think people from the various zoom meetings???
\subsection{Design Viewpoints}
\subsubsection{Context Viewpoint}
\subsubsection{Composition Viewpoint}
\subsubsection{Interface Viewpoint}
\subsection{Design Rationale}

%=======================================================
\section{Methods}
\subsection{User Interface Framework}
\subsubsection{Approach}
The uncertainty of a platform requires that the user interface framework is multi-platform.
The chosen framework, Qt, supports a variety of platforms with similar or identical interfaces with the underlying programming language.
The layout and bindings to underlying data will be written using Qt's provided API, and the required libraries will be packaged with the software as allowed by the license for distribution. 
\subsubsection{Concerns}

\subsection{Application Logic}
\subsubsection{Approach}

\subsubsection{Concerns}

\subsection{Interface Framework Integration}
\subsubsection{Approach}
Bound variables from Qt will be read by the underlying code.
These inputs will be checked for validity and converted into their proper data types.
Once converted into the types required for the application logic's calculations, the volumetric rate output is calculated and returned to the user.
The interface between the framework and the application logic will be performed using the Model-view-viewmodel design pattern.
\subsubsection{Concerns}
Software developers on open source projects focus their efforts highly on projects where they have the requisite skill.
Model-view-viewmodel is a common design pattern for UI integration in desktop and mobile applications. 
User Interface designers that wish to improve the design will find familiarity with web design frameworks, such as Vue.js, which uses a model-view-viewmodel method to interface between client and server.
Users will experience a lower performance impact, as the application will remain static when not in use.

%=======================================================
\section{Conclusion}
Will include in final rendition of paper. Still working out some stuff.
%=======================================================
\section{References}

“How to Store Data Locally in an Android App.” Android Authority, 20 Nov. 2017, www.androidauthority.com/how-to-store-data-locally-in-android-app-717190/.
\newline
\newline
“Top 10 Best Mobile App Development Frameworks in 2019–20.” By Alex Hales, hackernoon.com/top-10-best-mobile-app-development-frameworks-in-2019-612b95cf930f.
\newline
\newline
Krishna12. “How to GetCurrent Time without Using Picker.” Ionic Forum, 15 Nov. 2017, forum.ionicframework.com/t/how-to-getcurrent-time-without-using-picker/112208.
\newline
\newline
Zak-DevZak-Dev. “Cordova : Android Device Current Date/Time.” Stack Overflow, 1 Jan. 1968, stackoverflow.com/questions/46781636/cordova-android-device-current-date-time.

\end{document}
